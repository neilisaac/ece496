\section{Project Overview}

% The intent of this section is to re-introduce the reader to the project. It may be largely crafted
% from previous work if the project has not changed much from the proposal, although it may
% contain details of decisions made since the proposal. As it will set the stage for your progress
% descriptions, you might also want to include information about the main design challenges of
% your project. It must at least create the stage for the individual reports so the relevance of the
% individual work is apparent.
% It is expected that details pertinent to the individual progress will be in the documents from those
% individuals.
% If it makes sense, combine this section with the group progress summary. This section should
% NOT be as detailed as in your Proposal, and may use your Proposal as a reference.

The goal of this project is to provide a physical platform for researchers to carry out FPGA architecture studies.
We are developing an \emph{Overlay FPGA}\footnote{We also refer to the Overlay FPGA design as a \emph{Virtual FPGA}.}, an FPGA circuit working on top of a commercial FPGA product.
Our project will consist of the Overlay FPGA circuit, and software that enables the user to translate their Verilog code into a \emph{bitstream}\footnote{A bitstream is a file containing the encoded data required to program a circuit an FPGA.} that can be programmed onto the Overlay FPGA.
The goals of our project can be reviewed in Appendix C.

%% I don't think this paragraph is useful in this document %%

%We have selected the Xilinx Virtex 5 FPGA as our development platform because we can
%use its native logic units directly. This will limit the models of FPGA chips that the overlay
%circuit can be implemented on, but it should reduce the design's area overhead and improve
%its timing characteristics. The features we will use on the Virtex 5 are forward-compatible
%with all current-generation Xilinx FPGA products, allowing the researcher to use a variety
%of FPGAs.

The validation of the design will involve testing a set of benchmark circuits by placing and
routing them with VPR, then transferring them to the FPGA overlay. The circuits can
then be tested for correct behavior, confirming that the overlay design can be correctly programmed using VPR output, and that the inputs and outputs to the design are functioning
properly.


