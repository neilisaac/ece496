\section{Group Progress Summary}

% - A summary of the project goal and any changes to the goal or requirements since the
%   design review. A system-level diagram is often helpful. Be brief here since you will also
%   provide an updated copy of the complete project goal and requirements in Appendix B.
%
% - A summary of the group's progress. Highlight a few key accomplishments since the
%   design review. Briefly describe some of the key challenges that were encountered and
%   some of the key decisions that were made in this time. Is the group on schedule? Make
%   explicit reference to the milestones on the original Gantt chart from the Project Proposal.
%
% - The key responsibilities of each team member since the design review. One or two items
%   for each member is sufficient, and can be general areas instead of tasks.
%
% - A summary of any changes to the group work plan, individual responsibilities, or the
%   project milestones. Again, be brief here since you will provide the details in Appendix A.


%% timeline - nothing else related to project management has changed (goals, work plan, etc) - maybe we should say that explicitly. %%
Our project is progressing well and we are confident that we can meet our deadlines.
Our initial schedule was very optimistic, leaving a large amount of slack time at the end of the term.
We have adjusted the gantt chart to reflect more realistic expectations as we progress through the project and experience the true complexity of the individual tasks.
The updated gantt chart is included in Appendix A, and the old version is in Appendix B.
Aside from the schedule change, our goals, work plan, and work allocation has not changed.

%% tasks - could add our names to them to reflect bullet 3 above? %%
In order to build a working \emph{Overlay FPGA}, we need to produce the following:
\begin{itemize}
\item Verilog for the individual FPGA building-blocks: the logic cell, logic block, connection block, and the switch block. (Keyi)
\item Verilog to connect the building block components together in a grid and feed the programming signals through them. (Keyi)
\item Verilog circuit to interface with the UART to enable serial programming of the Overlay FPGA. (Neil)
\item Scripts to run the third-party tools which will process an input circuit (from the user) and produce a valid placment and routing. Some adjustments need to be made so the tools can read each others' outputs. (Neil)
\item Software to convert the placement and routing into a programmable bitstream. (Neil)
\end{itemize}


%% rest is status on tasks %%

The basic components of the overlay have been completed and are individually functional, but extensive debugging is required to get it all working together.

The software is also complete, but can't be fully tested until the hardware is fully functional.
Partial validation was done on the software by manually reading the bitstream file it produces.

Once we complete this debugging, we have a number of improvements planned to improve the quality of the virtual FPGA (in terms of usability to researchers, as well as its efficiency.)
These improvements are not necessary for a proof-of-concept, or for a fully functional demo.

We are confident that we can complete these remaining tasks before the design fair.



