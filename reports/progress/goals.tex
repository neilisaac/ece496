\section{Project Goals and Requirements}

% Insert a copy of the „Project Goal‟ and „Project Requirements‟ sections from the Project
% Proposal. If there are any changes, briefly justify the changes and provide an updated version of
% the current Project Goals and Requirements.

% I just copied this from the final proposal.  We may want to simplify it for the presentation.


\subsection{Functional Requirements}

An Overlay FGPA circuit that will:

\begin{itemlist}
	\item Work on a commercially available FPGA chip.
	\item Be re-programmable over a serial interface after being flashed to the FPGA.
	\item Have a tunable number and arrangement of logic cells, and have tunable connectivity parameters. 
	\item Support inputs to and outputs from test circuits programmed onto the Overlay FGPA.
\end{itemlist}

A software program that can:
\begin{itemlist}
	\item Translate VPR placement and routing data for a test circuit into a bitstream for the Overlay FGPA.
	\item Program the bitstream onto the Overlay FGPA to implement the circuit.
\end{itemlist}

\subsection{Constraints}

\begin{itemlist}
	\item The overlay circuit must support at least 3000 logic cells\footnote{3000 logic cells was chosen as the minimum target because the largest of the ``Golden 20'' circuits, \emph{``s38417''} requires 2567 6-input logic cells\cite{synthesis-density}.} in order to accommodate the \emph{``Golden 20''} MCNC benchmark circuits\footnote{The ``Golden 20'' MCNC circuits are available in BLIF format at \url{http://www.ece.ubc.ca/~julienl/benchmarks.htm}.} commonly used in FPGA research.
\end{itemlist}


\subsection{Objectives}

\begin{itemlist}
	\item Be compatible with a family of commercial FPGAs that are available to researchers.
	\item Use the native logic cells in the physical FPGA directly in the overlay FPGA design to reduce area and latency.
	\item Be fast enough that it outperforms software emulation of most test circuits.
\end{itemlist}

