\section{Introduction}

This report explains the motivation, implementation and testing, and limitations of our Overlay FPGA circuit.


\subsection{Background and Motivation}

Academic researchers who study Field Programmable Gate Array (FPGA) design commonly use variations of an FPGA design architecture, described by Kuon et al\cite{fpga}, which we will refer to as the \emph{Academic FPGA Model}.
While FPGA chips are available from a variety of commercial vendors, their inner design is proprietary, making their architectures difficult to study.
Furthermore, there is no existing physical implementation\footnote{Alex Brant is also developing a comparable FPGA overlay platform with Prof. Guy Lemieux at University of British Columbia.} of an Academic FPGA Model.

VPR\cite{vpr} is an open-source placement and routing tool used in FPGA architecture research.
It can handle many variations of the Academic FPGA Model.
It is not currently possible to implement circuits produced by VPR on a commercial FPGA.\footnote{A technology-mapped input netlist for VPR can be converted to an Altera Quartus VQM netlist file using \emph{nettovqm}\cite{nettovqm}, but the placement and routing can not be converted.}

Computer Aided Design (CAD) researchers who work on placement and routing algorithms for FPGA designs are presently limited to using simulations to evaluate or verify their work.
They may be interested in testing circuits on a physical medium because it would be much faster than simulating the circuits.



\subsection{Project Goal}

The goal of this project is to design a circuit design based on the Academic FPGA Model.
Researchers will be able to use the circuit to study FPGA architecture and CAD algorithms with circuits produced by VPR.



\subsubsection{Requirements}

Hardware (overlay) requirements:

\begin{tabular}{|p{1cm}|p{15cm}|}
\hline
H1 & Compatible with a commercially available FPGA chip\\ \hline
H2 & Capable of supporting user-specified number of logic cells and connectivity parameters \\ \hline
H3 & Re-programmable over a serial interface after being flashed to the FPGA \\ \hline
H4 & Support for inputs to and outputs from test circuits \\ \hline
\end{tabular}

%\begin{itemlist}
%	\item Work on a commercially available FPGA chip.
%	\item Be re-programmable over a serial interface after being flashed to the FPGA.
%	\item Have a tunable number and arrangement of logic cells, and have tunable connectivity parameters. 
%	\item Support inputs to and outputs from test circuits programmed onto the Overlay FGPA.
%\end{itemlist}

Software requirements:

\begin{tabular}{|p{1cm}|p{15cm}|}
\hline
S1 & Capable of translating VPR placement and route data for a test circuit into a bitstream for the Overlay FPGA\\ \hline
S2 & Capable of programming the bitstream onto the Overlay FPGA to implement the actual circuit \\ \hline
\end{tabular}

%\begin{itemlist}
%	\item Translate VPR placement and routing data for a test circuit into a bitstream for the Overlay FGPA.
%	\item Program the bitstream onto the Overlay FGPA to implement the circuit.
%\end{itemlist}

General requirements:

\begin{tabular}{|p{1cm}|p{15cm}|}
\hline
G1 & Support 6-input logic elements, which allows for comparison with current generation commercial FPGAs\\ \hline
G2 & Fit at least 100 overlay logic elements on the Virtex 5 FPGA \\ \hline
\end{tabular}

%\begin{itemlist}
%	\item Support 6-input logic elements, which allows for comparison with current generation commercial FPGAs.
%	\item Fit at least 100 overlay logic elements on the Virtex 5 FPGA.  This number was chosen to be adequately large for our proof of concent.  100 cells will fit our test circuits fit, and concevable circuits can be implemented by participants at the design fair.
	%\item The overlay circuit must support at least 3000 logic cells\footnote{3000 logic cells was chosen as the minimum target because the largest of the ``Golden 20'' circuits, \emph{``s38417''} requires 2567 6-input logic cells\cite{synthesis-density}.} in order to accommodate the \emph{``Golden 20''} MCNC benchmark circuits\footnote{The ``Golden 20'' MCNC circuits are available in BLIF format at \url{http://www.ece.ubc.ca/~julienl/benchmarks.htm}.} commonly used in FPGA research.
%\end{itemlist}


%\subsubsection{Objectives}

%\begin{itemlist}
%	\item Be compatible with a family of commercial FPGAs that are available to researchers.
%	\item Use the native logic cells in the physical FPGA directly in the overlay FPGA design to reduce area and latency.
%	\item Be fast enough that it outperforms software emulation of most test circuits.
%\end{itemlist}

