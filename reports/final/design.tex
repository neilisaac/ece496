\section{Final Design}


\subsection{System overview}
\label{system-overview}

\figref{system-diagram} shows the high-level system interactions of the components used in our project.
The components we are building shown in solid rectangles.

The finished project will consist of three main parts: The overlay FPGA, bitstream generation software, and interfacing of inputs/outputs of the overlay FPGA.

The \emph{Overlay FPGA} is a Verilog HDL circuit implementation of the Academic FPGA model which will be constructed as an overlay on a Xilinx FPGA board.
The arrangement, size and connectivity of the overlay circuit will be controllable via parameters in the source Verilog.
The overlay FPGA will consist of organized tiles of \emph{logic block}, \emph{connection block}, and \emph{switch block} modules.
Together, these modules will allow the overlay FPGA to implement different logic circuits.

\begin{figure}[!h]
	\centering
	\includegraphics[scale=0.6]{system.png}
	\caption{System overview}
	\label{system-diagram}
\end{figure}

Once built, the overlay FPGA can be configured to implement user-specified \emph{test circuits} that have been placed and routed by VPR.
This will be achieved by creating \emph{bitstream generation and programming software} that will translate the VPR output circuit into a bitstream that the overlay FPGA will understand.
The bitstream will then be injected into the overlay FPGA via a \emph{serial interface}.
The FPGA will receive and decode the bitstream into the appropriate test circuits on the overlay.

Finally, the circuits on the overlay FPGA can be tested for functionality by connecting devices (e.g. Switches and LEDs) to the \emph{Test circuit inputs} and \emph{Test circuit outputs} of the Overlay FPGA.
More elaborate input and output mechanisms will be explored once the Overlay FPGA is complete.



\subsection{Module design}

The \emph{Overlay FPGA} will be composed of a two-dimensional array of \emph{Logic tiles}.
The logic tiles make it easier to build a large overlay, and help keep the internal 
logic modules organized.
Each logic tile will consist of one \emph{logic block module}, two \emph{connection block modules}, 
and one \emph{switch block module}.
\figref{tile-diagram} shows the internal composition of a single logic tile.

\begin{figure}[!h]
	\centering
	\includegraphics[scale=0.7]{overlay.png}
	\caption{Logic tile and tile arrangement}
	\label{tile-diagram}
\end{figure}

The \emph{Logic block module} consists of programmable look-up tables that perform all of the logical 
funcionality required by the circuit.
A logic block module may be composed of multiple look-up tables, the number of which 
can be determined by verilog parameters.

The \emph{Connection block} and \emph{Switch block} modules regulate the routing of signals in the overlay.
\emph{Connection blocks} connect logic block signals to buses that run throughout the 
overlay.
\emph{Switch blocks} control the routing between buses when they cross each other.

The \emph{Bitstream generation software} will be a program that translates VPR output circuits into a 
bitstream capable of programming the overlay FPGA directly.
The program will consist of functions that parse the output from VPR and generate the 
appropriate bitstream from the parsed information.

The \emph{Bitstream programming software} will take the bitstream formed by the bitstream generator 
and format it for proper transmission over a serial interface.

The \emph{Serial decoder} will be a circuit attached to the overlay FPGA that receives the 
bitstream sent through the serial interface.
It will decode and extract the bitstream 
from the serial format, then inject it into the overlay circuit to configure the 
overlay.



\subsection{Assessment of design}

The decision to take advantage of the custom 32-bit shift registers in implementing our design entails the following trade-offs versus using only basic Verilog logic:
\begin{itemlist}
	\item The design is more efficient area and timing-wise.
	\item The data describing the circuit (known as the \emph{bitstream}) which is needed to program the design will be larger.
	\item The maximum size of the design will be limited by the amount of 32-bit shift registers available on the FPGA board, as opposed to the amount of flops.
	\item The design will be incompatible with boards that do not have custom 32-bit shift registers
\end{itemlist}

We have decided that performance efficiency outweighs the negative aspects of the larger bitstream and limitation of implementation platforms.
If the design is successful, adaptations can be made in the future to support the implementation of the design on more FPGA boards.


