\section*{Highlights for Keyi Shi}

My contributions to the final design were:
\begin{itemlist}
	\item shift multiplexing for all logic tile modules
	\item component modules for the overlay verilog design
	\item verilog overlay tile boundary design and optimization
\end{itemlist}

Multiplexers are used extensively in all our verilog modules.
Each module has different multiplexing requirements, and one of our design goals was to have a configurable size/routing parameters; as such, the sizes of multiplexers required by our design vary module to module, parameter to parameter.
I designed a flexible multiplexer capable of changing its size based on the number of inputs required, and this multiplexer is what drives all our overlay modules.

I focused on the routing elements of the overlay design, building the switch block and connection block modules and ensuring their compatibility with each other and the logic block.
The modules were all designed to be flexible, capable of accepting user parameters and changing their sizes and arrangements to fit user needs.
The modules also went through bitstream testing to make sure they behaved correctly under all use cases.

The verilog overlay tile combined basic verilog components into a single module, capable of being replicated in a grid formation to create the final overlay.
The boundaries of the overlay required special modules to tie the grid together and interface between the overlay's internal and input/output signals.
I wrote the boundary modules and debugged/optimized the overlay design, ensuring all signals between tiles are routed correctly, and the boundary signals are gathered into their appropriate input/output pads.
I tested the overlay with manual and generated bitstreams to debug both the overlay hardware and the bitstream software.