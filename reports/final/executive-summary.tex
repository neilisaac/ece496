\thispagestyle{empty}
\section*{Executive Summary}

% No more than 1 page
% Should show clear understanding of AUDIENCE and PURPOSE
% Readable as a stand-alone document, clearly differentiated from an introduction
% Give the contest and MOST IMPORTANT information in the document in a unified fashion

Academic studies of Field Programmable Gate Array (FPGA) chip architecture rely on simulations, as commercial FPGA chips contain proprietary designs that make their underlying architecture inaccessible to academic researchers.
The goal of this project is to provide a physical platform for researchers to carry out FPGA architecture studies.
The finished design is capable of implementing basic Verilog circuits with combinational and sequential logic.

Our design is an implementation of an Overlay FGPA on an existing, commercially available FPGA chip.
Using a commercial FPGA as the physical medium for this project makes the design cheaper and more accessible to researchers, as they may have an appropriate FPGA chip already.


We have selected the Xilinx Virtex 5 FPGA as our development platform because we can use its native logic units directly.
This will limit the models of FPGA chips that the overlay circuit can be implemented on, but it reduces the design's area overhead and improve its timing characteristics.
The features we use on the Virtex 5 are forward-compatible with all current-generation Xilinx FPGA products, allowing the researcher to use a variety of FPGAs.

The validation of the design involves testing a set of unit-test circuits we wrote by placing and routing them with VPR, then transferring them to the FPGA overlay.
The circuits can then be tested for correct behavior, confirming that the overlay design can be correctly programmed using VPR output, and that the inputs and outputs to the design are functioning properly.

The current budget for the proposed design was \$14.00 and was covered by the students.
The required FPGA development boards and software licenses have been provided by the supervisor.

