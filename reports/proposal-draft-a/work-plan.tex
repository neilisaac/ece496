\section{Work plan}

%\subsection{Work breakdown structure} % for Draft B

%\subsection{Gantt chart} % for Draft B

%\subsection{Financial plan} % for Draft B

\subsection{Feasibility Assessment}

% This section is meant to help your team, supervisor, and administrator assess the
% feasibility of your proposed project. This is not a marketing exercise: try to provide a fair
% and honest assessment of your proposal, balancing both its strengths and weaknesses.
% There is nothing wrong with identifying major deficiencies in your project; in some
% businesses, fewer than one in ten projects results in a commercially viable product.
% Identifying weaknesses and putting together a plan to address those weaknesses early on
% is a crucial part of the design process. It also helps your supervisor and administrator in
% their roles as effective mentors and guides.
% 
% Here are some of the key issues you should address in this section. Be brief - a sentence
% or two is probably enough to cover each issue. Also, these issues do not need to be
% addressed in any particular order and may be combined or reorganized to flow logically:
% Skills and resources:
% - What are the key skills, knowledge and resources you need for this project?
% - What portions have you already acquired and what portions are currently lacking?
%   How do you plan to obtain what you still need? Examples:
%    o From the web: free or open source software, technical standards, expert
%      forums
%    o From your supervisor: graduate researchers, lab space and equipment, etc.
%    o From the Design Centre (SFB520): facilities for making and soldering
%      printed circuit boards, used hardware from past design projects,
%      computers, test equipment, etc.
%
% Risk Assessment:
% In this section, describe the risks that the project could face (risk identification) and how
% you plan to deal with them (risk mitigation). An example of a risk would be that a
% particular component you envision might prove impossible to implement, and a
% corresponding risk mitigation strategy would be attempting to prototype the riskiest
% portion of your project as a feasibility assessment of the whole project. This would be
% coupled with a „fall-back‟ plan as to what you would do should this prototype fail.
% Merely, stating that risks will not occur, or that risks will be mitigated by working harder
% does NOT constitute back-up plan!
% This section should not be very long and you should focus on one or two real risks to
% your project (most likely technical risks but there could be others as well). Minor risks
% that will cause little disruption in schedule do not need to be addressed. Note that
% something that is technically challenging (or difficult to implement) may not necessarily
% be a significant risk to the project, if its failure does not affect the overall project goals or
% requirements (for example, the task may relate to a project objective for an additional
% feature, rather than a core requirement).
% Note: when addressing risks, focus on the most likely ones that are specific to your
% project. Some students ponder the risks of having a team member drop out of school, or
% losing all their work due to a computer crash. Such discussions aren't particularly helpful
% in planning your project. A more specific and effective series of questions might be
% - What happens to other components and to the project if component X that I‟ve
%     designed for my system fails to meet specifications or takes significantly longer to
%    develop?
% - Can I change the specifications, demonstrate a lower performance system, or
%   remove some features of my final design and still maintain the essential aspects of
%  my project?
% - What if our initial plan to build a real prototype proves unfeasible? Could we
%   demonstrate our design or part of our design using a computer model instead?
%  What would the limitations of this computer model-based prototype be? How
% would the project goal, requirements, and scope be modified to ensure that the
% new, redefined, project remains challenging?
% The key idea is to think of ways that you can modify the scope of your project so that you
% can show some partial success in realizing your Project Proposal by the end of the school
% year. More information and additional examples in [Design Notes, Chapter 11].
%
% Risk in Research Projects: Often, a research project will have above-average risk and
% this section may need to be longer. If the „Technical Design‟ section describes your
% intended initial strategy, this section should describe alternate directions that will be
% taken if experimental results indicate that the initial strategy is no longer what should be
% done. Here, the „risk‟ is that the experimental result or some intermediate result not be as
% expected (and the probability of this could be high) and the mitigation is the
% determination of an the alternate goal or an alternate route to the original goal.





