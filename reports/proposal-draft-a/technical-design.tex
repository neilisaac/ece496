\section{Technical Design}

\subsection{Design Alternatives}

% In this section, you explore and discuss different possible solutions and design
% alternatives. Exploring possibilities is often neglected by designers eager to start on the
% first idea that comes to mind. Often, however, the first solution isn‟t the best. For
% instance, you may have in mind an implementation using a keyboard, but when you work
% back to the requirements you may realize that it is only the user control aspect that is
% required, and thus you can do it all from the attached personal computer. The key to
% designing is coming up with alternatives, and it is in exploring alternatives that you come
% to appreciate the inevitable design trade-offs that you will face.
% 
% DO NOT ESTABLISH A DESIGN CHOICE, AND THEN THINK ABOUT
% ALTERNATIVES JUST TO GET THIS DOCUMENT DONE.
% 
% Some alternatives may differ only in small variations in implementation, others may be
% quite different. You should provide enough of an evaluation of each choice to justify your
% selection of the proposed solution. Provide a preliminary assessment of the different
% design alternatives in terms of the project goal and requirements you've laid out. Create a
% comparison table if necessary [See Design Notes, Chapter 9 for more ideas].
% 
% You may find that this section and the next naturally collapse into a single section, or that
% you wish to keep them separate.



%\subsection{Assessment of Proposed Design} % for Draft B

%\subsection{System-level overview} % for Draft C

%\subsection{Module-level descriptions} % for final proposal

